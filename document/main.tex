\documentclass[%
 aapm,
 mph,%
 amsmath,amssymb,
%preprint,%
 reprint,%
%author-year,%
%author-numerical,%
]{revtex4-2}

\usepackage{graphicx}% Include figure files
\usepackage{dcolumn}% Align table columns on decimal point
\usepackage{bm}% bold math

\usepackage{mathtools}
\usepackage[mathlines]{lineno}% Enable numbering of text and display math

\begin{document}

\preprint{AAPM/123-QED}

\title[Ion transport in a retarding field energy analyzer]{ERIKA: Emulating Retarding-field-energy-analyzer Ion Kinetic-transport in Argon-gas}

\author{Felipe Soberon}
 \altaffiliation{Impedans Ltd.}
 \email{felipe.soberon@impedans.com}
 \homepage{http://www.impedans.com}

\date{\today}

\begin{abstract}
This document reports a one-dimensional simulation of the transport of argon ions across a plasma sheath and through a
retarding field energy analyzer (RFEA). The simulation can model DC and AC sheaths. The DC sheath is modeled using Child Law
and the AC sheath using the analytical solution of a radio-frequency capacitive sheath by M. A. Lieberman. The model includes
elastic and charge exchange ion collisions with background Argon gas. 
\end{abstract}

\keywords{Plasma sheath, Child Law, RF sheath, Charge exchange collision, Retarding Field Energy Analyzer (RFEA)}
\maketitle

\section{\label{Introduction}Introduction}

Retarding field analyzers are employed to assess the energy distribution of ions impinging on a surface. Ions acquire considerable kinetic energy as a result of the high voltage across the plasma sheath. This energy distribution is valuable in various applications, such as space propulsion equipment. In other contexts, such as the manufacturing of semiconductor devices, plasma etching equipment, or plasma deposition, the ion energy serves as a parameter that can influence the etch rate, the profile of the process, or the deposition rate.

RFEA devices can be designed as an array of thin grids tightly packed together, with a series of voltages applied to these grids. The primary purpose of this configuration is to select ions based on their energy and to eliminate the detection of electrons originating from the plasma discharge or secondary electrons emitted by the RFEA following ion impact. 

The operational range of RFEA devices is limited by several factors, including background gas pressure, fast electrons from the plasma, secondary electrons emitted within the RFEA, ionization occurring inside the RFEA, the distance between RFEA grids, and field distortion within the RFEA caused by space charge.

To assess the impact of any of the factors mentioned above, a computer simulation of a plasma sheath and RFEA grid would enable one to explore these factors individually or in combination.

Here we report a one-dimensional model of a plasma sheath and RFEA device. 
 
\section{\label{Simulation}Simulation Description}
In this section we provide details of the model. Subsection \ref{SpacePotentialAndField} describe how the space potential and electric field are calculated including the plasma sheath and in the RFEA region. Subsection \ref{IonTrajectory} describe the method to integrate the ion trajectories under the effect of the fields decribed in subsection \ref{SpacePotentialAndField}. Subsection \ref{IonCollision} describes the method to simulate ion collisions with the background gas. 

\subsection{\label{SpacePotentialAndField}Space Potential and Electric Field}
The space potential and electric field are calculated in two regions: the RFEA region, and the plasma sheath. The model is one-dimensional, therefore only space potential and electric field in the z-axis is calculated in these regions. In this geometry, the RFEA is located at $z \leq 0$ and its extension in the negative direction of the z-axis depends on the spaces between the grids and the collector; subsection \ref{RFEA}. The plasma sheath starts at $z=0$ and extends to the plasma, which is located at some $z_\text{plasma} > 0$. The location of the plasma sheath boundary edge depends on the plasma sheath model and is further described in subsection \ref{SheathModel}.   

\subsubsection{\label{RFEA}Retarding Field Energy Analyzer}
The RFEA modeled includes the standard four grids and collector plate~\cite{Hutchinson1987}. These are identified as G$_0$, G$_1$, G$_2$, G$_3$, C. The first grid, G$_0$, faces the plasma and is at the same potential as the electrode that bounds the plasma discharge. In this model, the voltage of G$_0$ is set to zero. The second grid, G$_1$ is typically biased negative respect to G$_0$ to repel electrons from the plasma discharge. The model does not include electrons, however, this grid is included to simulate the electric field that affects the motion of ions between G$_0$ and G$_1$. The third grid, G$_2$, is used to discriminate ions by setting the voltage of G$_2$ positive relative to G$_0$. The voltage of G$_2$ is typically swept from zero to a large voltage to limit the ion current to the collector and generate the RFEA characteristic voltage-current curve. The fourth grid, G$_3$, is used to repel secondary electrons that may be emitted by the collector due to ion impact. This grid is usually biased slightly negative relative to the voltage applied to the collector. Again, the model does not include electrons, however, it is included to simulate the electric field between G$_3$ and the collector that afect the ion trajectories. Finally, the last electrode in the RFEA is the collector, C, which is biased somewhat negative relative to the first grid, G$_0$.

The model assumes that the grids and non-dimensional; i.e. zero thickenss. This differs from actual RFEA devices where the grids are usually 40~$\mu$m thick (reference some publication by Impedans). Consequently, the model also neglects any lensing effects of non-zero thick grids~\cite{vandeVen2018,Buiter2018}. The distance between grids and the collector are defined in the model as multiples of 100~$\mu$m. In actual RFEA devices the grids are physically separated by sheets of Mica, 100$\mu$m thick each sheet. The arrangement of grids and spacers is known as the RFEA button stack. For instance, the standard RFEA button stack of a commercial device comprises two spacers between G$_0$ and G$_1$, three spacers between G$_1$ and G$_2$, three spacers between G$_2$ and G$_3$, and two spacers between G$_3$ and the collector C (reference impedans website Semion). This arrangement is therefore known as a 2332 button stack. The model can simulate the RFEA with any combination of button stack arragements expressed as multiples of the basic spacer. 

Besides lensing effects, not included in this model, actual grids are not entirely transparent and a fraction of ions moving through the RFEA are collected by the grids. The transparency of the grids depends on the ion energy and may vary due to the lensing effect. While the model presented here does not simulate these effects it includes a simple transparency factor setting. The user can set this number to any value from 0 to 100\%. The transparency setting is common to all grids; i.e., the transparency of grids cannot be set individually.     

The space potential between grids is calculated by linear interpolation. Similarly, the electric field between grids is calculated as the potential difference between each grid pair divided by the distance. Figure~\ref{fig:DCpotentialField} shows the space potential and electric field in the RFEA for grid voltage settings G$_0$=0, G$_1$=-60, G$_2$=250, G$_3$=-70, and C=-60~V. The location of the grids and collector is indicated with vertical dashed lines. 


\subsubsection{\label{SheathModel}Sheath Model}
The space potential and electric field in the sheath region of the plasma is determined from the Child Law sheath model for direct-current sheath~\cite{Lieberman2005}, and from the analytical solution of capacitive radio-frequency sheath for alternate-current sheath~\cite{Lieberman1988}. 

The Child Law sheath size is calculated 
\begin{equation}
s_\text{Child Law} = \frac{\sqrt{2}}{3} \lambda_D \left( \frac{2 V_0}{T_e} \right)^{3/4}
\end{equation}
Where $\lambda_D$ is the Debye length
\begin{equation}
\lambda_D = \sqrt{ \frac{\epsilon_0 T_e}{e n_s} }
\end{equation}
Where $T_e$ is the electron temperature (in eV), $\epsilon_0$ is vacuum permittivity (in Farads/metre), $e$ is the basic electric charge (in Coulomb), and $n_s$ is the plasma density at the sheath edge (in m$^{-3}$).

The voltage in the Child Law sheath is given by
\begin{equation}
V_\text{Child Law}(z) = V_0 \left( 1 -  \left( 1 -\frac{z}{s_\text{Child Law}} \right)^{4/3} \right)
\end{equation}
Where the position variable, $z$, is given in metres, and where $V_0$ is the voltage of the plasma sheath edge relative to the bounding electrode at $z=0$. The voltage is null at $z=0$ and $V_0$ at $z=s_\text{Child Law}$. The plasma potential is positive relative to the electrode, therefore $V_0$ is positive.

The electric field (in V/m) is given by
\begin{equation}
E_\text{Child Law}(z) = -\frac{4}{3} \frac{V_0}{s_\text{Child Law}} \left( 1 - \frac{z}{s_\text{Child Law}}  \right)^{1/3}
\end{equation}
Note the field is null at $z=s_\text{Child Law}$ and negative otherwise in the sheath region. 

Figure~\ref{fig:DCpotentialField} shows the space potential and electric field for the DC sheath solution with a voltage across the sheath of 1000~V and plasma density $n_s = 10^{16}$~m$^{-3}$. Note that even for a small voltage at the discriminator grid G$_2$ (relative to the plasma potential) the electric fields within the RFEA are much higher than the electic field across the sheath due to the proximity between grids; i.e., ions experience substantially higher forces within the RFEA. 

\begin{figure}[htbp]
\centering
\includegraphics[width=0.45\textwidth]{Figures/VEz2DC1kVStack2332.jpeg}
\caption{Space potential across a plasma sheath and RFEA grid stack (top) and corresponding electric field (bottom) for DC plasma sheath with a voltage across the sheath of 1000~V. The edge of the plasma sheath and plasma is at $z=0.8$~cm. The RFEA is located at $z\le0$; with the plasma facing grid (G$_0$) at $z=0$. The location of the RFEA grids and collector are indicated by vertical dashed lines.}
\label{fig:DCpotentialField}
\end{figure}



The space potential and electric field for the AC sheath are calculated using the capacitive sheath solution of reference~\cite{Lieberman1988}. The sheath edge of the capacitive radio-frequency discharge changes through the cycle, from collapsing when the voltage across the sheath drops to its minimum to full expansion when the voltage is maximum. The size of the sheath is given by, 
\begin{eqnarray}
s(\phi) &=& s_0 ( 1 - \cos(\phi) + \frac{H}{8} \{ \frac32 \sin(\phi)  + \frac{11}{18} \sin(3 \phi) \nonumber \\ 
        & & - 3 \phi \cos(\phi)  - \frac13 \phi \cos(3 \phi)  \} )
\end{eqnarray}
With $\phi = \omega t$ ($\omega = 2 \pi f$) and $f$ the radio-frequency. The value of $s(t)$ is limited to $\phi$ in the range from 0 ($t=0$) to $\pi$ ($t=T/2$; where $T=1/f$ is the radio-frequency cycle period). The size of the sheath at $t=0$ is zero and its maximum size at $s(t=T/2) = s_m$. The $s(t)$ function can be evaluated at any time value via the following 
\begin{eqnarray}
s(t)    &=& \text{abs}(s(\omega \tau(t))) \\
\tau(t) &=& \text{remainder}\left(\frac{t}{T}+\frac12\right) - \frac{T}{2} 
\end{eqnarray}
The parameter $s_0$ is given by 
\begin{equation}
s_0 = \frac{J}{e n_s \omega}
\end{equation}
$J$ is the total current through the plasma discharge (mainly displacement current in the sheaths and conduction current in the plasma bulk). The current can be calculated from the plasma density at the sheath edge ($n_s$) and the maximum voltage across the sheath ($V_0$)
\begin{equation}
J = \frac25 \omega \sqrt{\frac65} \sqrt{e n_s \epsilon_0 (\sqrt{576+125 V_0}-24)}
\end{equation}
The $H$ parameter is given by
\begin{equation}
H = \frac1{\pi} \frac{s_0^2}{\lambda_D^2}
\end{equation}

The electric field in the z-direction is given by 
\begin{equation}
E_{AC}(z,t) = \frac{J}{\epsilon_0 \omega} \left \{ \cos(\omega t) - \cos(\phi(s_m - z)) \right \} 
\end{equation}
If $s(t)<s_m-z$, otherwise the electric field is zero. Note that the function $\phi(s)$ is the inverse of the sheath size $s(\phi)$, that it does not have an analytical solution, and that it is numerically solved in the computer code. 

Finally, the space potential for the capacitive radio-frequency sheath is given by the integral
\begin{equation}
V_{AC}(z,t) = \int_{0}^{z} E_{AC}(z\prime, t) dz\prime
\end{equation}
With $z$ in the range from $z=0$ to $z=s_m$. The field is numerically integrated.  




\subsection{\label{IonTrajectory}Ion trajectory integration}

\subsection{\label{IonCollision}Collisions of ions with background gas}


\section{Results and Discussion}
In this section we present results of runing the simulation described in the previous section. The subsections below focus on the following research questions, 

\begin{itemize}
  \item How well does the RFEA derived ion energy distribution function represents the actual energy distribution function of ions at the bounding electrode (i.e., at G$_0$)? 
  \item What is the effect of ion space charge inside the RFEA on the space potential and electric field?  
  \item What is the effect of different spacings between grids on the RFEA derived ion energy distribution function when compared to the actual one at the bounding electrode?
  \item How does the pressure transparency of the RFEA changes with pressure? For any given pressure, does the pressure transparency change with the discrimination, G$_2$, voltage? 
\end{itemize}


\subsection{Ion energy distribution profile}
Fisrt, we run the simulation at various pressures and compared the derived ion energy disribution function with the actual energy distribution as recorded in the model at G$_0$. The RFEA configuration is as follows: 2332 stack, G$_1$=C=-60~V, G$_3$=-70~V, grid transparency 100~\%. The plasma sheath potential used is AC with maximum voltage difference across the sheath of 2000~V (i.e., V$_\text{pk-pk}$), the radio-frequency 13.56~MHz, and the plasma density $10^{16}$~m$^{-3}$. The pressure settings are 0.5, 1.0, 2.0, 5.0, 7.5 and 10.0~Pa. The model is run sweeping grid G$_2$ voltage setting from 0 to 1500~V in steps of 25~V. The number of ions whose trajectory is simulated per G$_2$ voltage step is 25000. The starting time for each ion is randomized uniformly across the radio-frequency period to represent that ions may enter the sheath at any point during the RF cycle. Each ion trajectory is followed for a total time of 1~$\mu$s. 

Figure~\ref{fig:IVcurve} shows the characteristic current-voltage curve for each model RFEA scan. The vertical axis is the ion count at the collector position. The ion count drops as the pressure is increased. 

\begin{figure}[htbp]
\centering
\includegraphics[width=0.45\textwidth]{Figures/IVcurve.jpeg}
\caption{Current-voltage characteristic curves where the current is represented by the ion count collected at the collector position. The curves from highest ion count value to lowest correspond to pressure settings from 0.5 to 10.0~Pa.}
\label{fig:IVcurve}
\end{figure}

The first derivative of the ion current is proportional to the ion energy distribution function~\cite{Hutchinson1987}. The set of curves on the left of Figure~\ref{fig:PressureScan} shows the first derivative of the curves in Figure~\ref{fig:IVcurve}. The derivatives are normalized and plotted with an offset in the vertical axis from lowest pressure setting at the top to highest pressure setting at the bottom. The ion energy distributions derived from the ion counts at the collector in the simulated RFEA for the various pressure settings can be contrasted with the actual ion energy distributions recorded at G$_0$, before the ions transport through the RFEA. These IEDFs are shown on the right of Figure~\ref{fig:PressureScan}. These scan was repeated using a smaller integration time step for the ion trajectories, $dt=10$~ps, to assess any potential artifact introduced in the simulation by the trajectory integration method, and no substantial qualitative difference on the energy distribution functions was observed (not shown). The derived IEDF at the collector exhibits more noise than the IEDF at G$_0$ due to the nature of the numerical derivative of the collector ion count; i.e., numerical derivatives of any parameter always ampiflies the parameter's noise. Here, a centered finite difference method was used to differentiate the Collector ion counts. All curves are shown without any smoothing. Still, both distributions show similar features, which is expected as ion-neutral collisions can not affect the shape of the ion energy distribution function~\cite{Baloniak2010}.   

\begin{figure}[htbp]
\centering
\includegraphics[width=0.45\textwidth]{Figures/PressureScan.jpeg}
\caption{Ion energy distribution functions. Left: First derivative of the current-voltage characteristic curves of Figure~\ref{fig:IVcurve}. Right: Plot of the ion energy distribution as recorded in the simulation at G$_0$, before ion transport through the RFEA. Both set of curves are normalized. The curves from top to bottom correspond to pressure settings from 0.5 to 10.0~Pa.}
\label{fig:PressureScan}
\end{figure}

The ion energy distributions of Figure~\ref{fig:PressureScan} exhibit a series of peaks which are the result of the oscillating plasma sheath and ion collisions with the background gas. More specifically, the sheath modulation, i.e., the time varying sheath edge and sheath electric field, can shape specific features of the IEDFs~\cite{Wild1989,Wild1991}. 




Second, we run the simulation at various radio-frequencies and compared the derived ion energy disribution function with the actual energy distribution as recorded in the model at G$_0$. The RFEA configuration is the same as the previous scan including the sheath potential maximum voltage difference 2000~V (i.e., V$_\text{pk-pk}$), the plasma density $10^{16}$~m$^{-3}$, and fixed pressure setting of 0.5~Pa. The model is run sweeping grid G$_2$ voltage setting from 0 to 1750~V in steps of 25~V. The number of ions whose trajectory is simulated per G$_2$ voltage step is 25000. The starting time for each ion is randomized uniformly across the radio-frequency period and each ion trajectory is followed for a total time of 1~$\mu$s. 

Figure~\ref{fig:IVcurve_FrequencyScan} shows the characteristic current-voltage curve for each model RFEA radio-frequency scan. The vertical axis is the ion count at the collector position. The set of curves on the left of Figure~\ref{fig:FrequencyScan} show the first derivative of the curves in Figure~\ref{fig:IVcurve_FrequencyScan}. The derivatives are normalized and plotted with an offset in the vertical axis from lowest radio-frequency setting at the top to highest at the bottom. The IEDFs recorded at G$_0$ are shown on the right of Figure~\ref{fig:FrequencyScan}. The distribution functions show a wider energy gap (between the lower energy peak to the higher energy peak) as the radio-frequency driving the sheath is reduced~\cite{Lieberman2005}. The lower frequencies have a stronger modulation effect on ions in transit through the sheath. However, there is no shift in the average of the energy peaks (mid point between the two peaks) as reported by Gahan et al.~\cite{Gahan2008}. Moreover, everything else being equal, as the radio-frequency is increased the maximum ion energy is approximately half the maximum voltage across the sheath; i.e., at 27.12~MHz, a 2~kV$_{pk-pk}$ sheath modulation accelerates ions up to approximately 1~keV.  

\begin{figure}[htbp]
\centering
\includegraphics[width=0.45\textwidth]{Figures/IVcurve_FrequencyScan.jpeg}
\caption{Current-voltage characteristic curves where the current is represented by the ion count collected at the collector position. The curves correspond to frequency settings from 2~Mhz to 27.12~MHz. The pressure was set to 0.5~Pa. }
\label{fig:IVcurve_FrequencyScan}
\end{figure}

\begin{figure}[htbp]
\centering
\includegraphics[width=0.45\textwidth]{Figures/FrequencyScan.jpeg}
\caption{Ion energy distribution functions. Left: First derivative of the current-voltage characteristic curves of Figure~\ref{fig:IVcurve}. Right: Plot of the ion energy distribution as recorded in the simulation at G$_0$, before ion transport through the RFEA. Both set of curves are normalized. The curves from top to bottom correspond to frequency settings from 2~MHz to 27.12~MHz. The pressure was set to 0.5~Pa.}
\label{fig:FrequencyScan}
\end{figure}





\subsection{Space charge inside the RFEA}
In high density plasmas, the ion current may be sufficiently high to allow for the build up of space charge within the RFEA~\cite{Hutchinson1987}. The charge build up may be highest for G$_2$ voltages at which most of the ion current is cut off. The model is used by simulating multiple ion trajectories through the plasma sheath and the RFEA. The trajectories are run for a maximum time of 1~$\mu$s. For each trajectory a random number is generated between 0 and 1 and is multiplied by the maximum time. This simulates the condition that not all ions depart the plasma at the same time and allow for spatial charge build up from the plasma sheath. The grid transparency is set to 50~\% to simulate ions collected by the grids and therefore removed from the space charge build up. The final position of the ions for each trajectory, if not collected by a grid or the collector, is recorded and a space charge density can be plotted. This simulation can be repeated for various G$_2$ voltages, pressures, sheath voltage, and radio-frequency.  






\subsection{Pressure transparency estimate}
The ion current to the collector in an RFEA can be attenuated mainly by two factors: a geometrical transparency factor, and a collisional transparency factor.  

First, each grid in the analyzer has a open area which can be used to estimate a geometrical transparency. Typically, the open area is 50~\%. The geometrical transparency is influenced by variations in the dimensions of grids openings, the number of openings in each grid, the alignment between grids and lensing effects due to the electric field distortion around the grid surface. As discussed in the description section, we can assign a geometrical transparency to every grid. The geometrical transparency (P$_g$) is the fraction of ions that may reach the collector where the impediment is only due to geometrical effects.  

Second, even though the spacing between grids is typically small (i.e., in the order of 100s of $\mu$m) ion collisions with neutrals may occur even at low pressures. As described before, the dominant collision process is charge exchange which effectively takes away all the energy gained by the ion in the sheath. Therefore, any collisions before the discriminator G$_2$ or at a distance after the discriminator grid where the space potential is below the potential of the collector would prevent said ion from reaching the collector~\cite{Baloniak2010}. The collisional transparency (P$_c$) is the fraction of ions that may reach the collector where the impediment is only due to ion collisions with the neutral background gas. 

The total ion current attenuation is the product of the two transparency factors.
\begin{equation}
P_{\text{total}} = P_\text{g} + P_\text{c}
\end{equation}

The collisional transparency ($P_c$) can be estimated from a theoretical approach as follows~\cite{Baloniak2010}, 
\begin{equation}
P_c = e^{- L / \lambda_i} 
\end{equation}
Where $L$ is the size of the RFEA stack, and $\lambda_i$ is the ion mean free path. The ion mean free path is
\begin{equation}
\lambda_i = (n_g \sigma)^{-1}
\end{equation}
Where $n_g$ is the density of the neutral background gas (in m$^{-3}$), and $\sigma$ is the collision cross section for ion-neutral collisions (in m$^2$). Note that the collision cross section is a function of the ion energy; see subsection~\ref{IonCollision}. However, for the purpose of a simple approximation the cross section can be taken as a constant value. Such simple approximation is used in the sheath model to make adjustments to account for ion collisions; see equation~\ref{eq:lambda_i}. The gas density can be estimated from the ideal gas law, 
\begin{equation}
n_g = \frac{p}{k_B T_g}
\end{equation}    
Where $p$ is the pressure of the background gas in Pascal, $k_B$ is the Boltzmann constant, and $T_g$ is the temperature of the background gas in Kelvin. Using this expression for the gas density we can estimate the collisional transparency as
\begin{equation}
P_c = e^{- L \sigma p / k_B T_g}
\end{equation}
Alternatively, using equation~\ref{eq:lambda_i}, the collisional transparency can be expressed as
\begin{equation}\label{eq:CollisionalTrasparency}
P_c = e^{- 30 L p / 0.13332237}
\end{equation}
By comparing the last two equations we can determine that if gas temperature is set to 300~K the collisional cross section used in equation~\ref{eq:lambda_i} is $\sigma \approx 93.2 \times 10^{-20}$~m$^2$. However, the cross section data shown in Figure~\ref{fig:CrossSectionsArgon} indicates that such value corresponds to ions with energies well below 1~eV. The choice of cross section value in this approximation of the collisional transparency has a strong influence in the outcome of the calculation.  

The pressure scan results can be used to estimate the collisional transparency. Using a similar procedure as that done experimentally, we can compare the number of ions reaching C with those passing through G$_0$ with no ion discrimination, i.e., G$_2 =0$. In the simulation, for each G$_2$ voltage setting 25000 ion trajectories were simulated. Therefore, the number of ions reaching the collector when G$_2$ is set to null divided by the 25000 is the collisional transparency. Figure~\ref{fig:CollisionalTransparency} shows the estimated collisional transparency from the pressure scan data between 0 and 10~Pa. Note that the collisionless sheath model is used for pressure 0~Pa. This sheath solution overestimates the discharge current and underestimates the sheath size. The collisionless sheath solution is closer to the collisional solution at pressure 0.5~Pa. 

Figure~\ref{fig:CollisionalTransparency} shows data from scans with a 2332 and 2772 button stack. The theoretical collisional transparency of equation~\ref{eq:CollisionalTrasparency} strongly underestimates the value. Using a collision cross section value of $\sigma \approx 22.5 \times 10^{-20}$~m$^2$ results in better agreement. This cross section value is closer to the cross section corresponding to ions with energies around 1~keV which is the upper limit energy for the ions in these simulation for the lowest pressures. As the pressure is increased the mean energy is lowered. But as the pressure increases this lower value of cross section results in higher collisional transparencies, which underestimate the effect of collisions reported in the model. This discrepancy suggests that the effect of collisions is therefore more complex than the theoretical formulation.  

The data shown in Figure~\ref{fig:CollisionalTransparency} is fitted to an exponential decay model. The fitted curves, with R$^2 > 0.99$, are
\begin{eqnarray}
P_\text{c:2332} (p) &=& e^{-0.0666 p} \\
P_\text{c:2772} (p) &=& e^{-0.105 p} 
\end{eqnarray}    

\begin{figure}[htbp]
\centering
\includegraphics[width=0.45\textwidth]{Figures/CollisionalTransparency.jpeg}
\caption{Estimated collisional transparency (P$_g$) as a function of argon neutral gas pressure. The RFEA button stack configurations are: 2332:triangles, and 2772:cicles. The data is tabulated in Table~\ref{table:CollisionalTransparency}.}
\label{fig:CollisionalTransparency}
\end{figure}



\begin{table*}
\centering
\caption{Collisional transparency, discharge current, ion mean free path and sheath size as a function of pressure from simulations with RFEA button stacks 2332 and 2772. The collisional transparency data is plotted in Figure~\ref{fig:CollisionalTransparency}. Note that the solution at pressure 0~Pa uses the collisionless sheath solution and that it underestimates the sheath size and overestimates the discharge current when compared with the 0.01~Pa results. The 0~Pa solultion is closer to the 0.5~Pa result. }
\label{table:CollisionalTransparency}
\begin{tabular}{c|cc|cccc}
\hline
Pressure (Pa) & \multicolumn{2}{c}{Collisional Transparency (Pc)} & Discharge Current (A/m$^2$) & Ion mean free path (cm) & Sheath size (cm) \\
              & 2332  & 2772 &                                &                      &                \\ \hline
0             & 1.00  & 1.00 & 97.1                           & N/A                  & 1.05           \\ \hline
0.01          & 0.98  & 0.98 & 47.6                           & 44.44                & 2.05           \\
0.5           & 0.97  & 0.96 & 100.7                          & 0.89                 & 1.01           \\
1             & 0.95  & 0.91 & 114.2                          & 0.44                 & 0.91           \\
2             & 0.90  & 0.84 & 129.1                          & 0.22                 & 0.82           \\
5             & 0.74  & 0.62 & 150.6                          & 0.09                 & 0.72           \\
7.5           & 0.62  & 0.46 & 160.7                          & 0.06                 & 0.69           \\
10            & 0.50  & 0.33 & 168.1                          & 0.04                 & 0.66           \\ \hline
\end{tabular}
\end{table*}


The collisionless sheath model underestimate of the sheath size results in discrepancies in the IEDFs at low pressures. The effect of a smaller sheath is that the ion energy peaks appear to be separated a wider distance than when the sheath is larger, as in the case of 0.01~Pa. Figure~\ref{fig:IVcurves2332} and Figure~\ref{fig:IVcurves2772} show the IV curves for the two different button stacks at pressures, 0, 0.01, 0.5, 1, 2, 5, 7.5 and 10~Pa. Figure~\ref{fig:PressureScan2332} and Figure~\ref{fig:PressureScan2772} show the IEDF curves for the two different button stacks at pressures, 0, 0.01, 0.5, 1, 2, 5, 7.5 and 10~Pa. The energy separation in the IEDF peaks is estimated~\cite{Gahan2008} as 
\begin{equation}
\Delta E = \frac{2 e V_0}{ \pi } \frac{ \tau_\text{rf} }{ \tau_\text{i} }
\end{equation} 
Where $V_0$ is the peak to peak radio-frequency modulation voltage across the sheath, $\tau_\text{rf}$ is the radio-frequency period, and $\tau_\text{i}$ is the ion transit time across the sheath. Naturally, a smaller sheath translates to a shorter ion transit time and therefore a larger energy difference $\Delta E$. In other words, the larger the sheath, the longer the ion transit time, and therefore the smaller the peak energy difference. In the collisional model, the lower the pressure, the larger the sheath and therefore the longer the ion transit time and the smaller the peak ion energy difference.  

\begin{figure}[htbp]
\centering
\includegraphics[width=0.45\textwidth]{Figures/IVcurve2332.jpeg}
\caption{IV curves for pressure scan with conditions listed in Table~\ref{table:CollisionalTransparency} with the RFEA set with a 2332 stack.}
\label{fig:IVcurves2332}
\end{figure}

\begin{figure}[htbp]
\centering
\includegraphics[width=0.45\textwidth]{Figures/PressureScan2332.jpeg}
\caption{IEDF curves for pressure scan with conditions listed in Table~\ref{table:CollisionalTransparency} with the RFEA set with a 2332 stack.}
\label{fig:PressureScan2332}
\end{figure}

\begin{figure}[htbp]
\centering
\includegraphics[width=0.45\textwidth]{Figures/IVcurve2772.jpeg}
\caption{IV curves for pressure scan with conditions listed in Table~\ref{table:CollisionalTransparency} with the RFEA set with a 2772 stack.}
\label{fig:IVcurves2772}
\end{figure}

\begin{figure}[htbp]
\centering
\includegraphics[width=0.45\textwidth]{Figures/PressureScan2772.jpeg}
\caption{IEDF curves for pressure scan with conditions listed in Table~\ref{table:CollisionalTransparency} with the RFEA set with a 2772 stack.}
\label{fig:PressureScan2772}
\end{figure}

\section{\label{Conclusion}Conclusion}

Here we have presented a model of the transport of ions across a plasma sheath and through a one-dimensional retarding field energy analyser. We have simulated the plasma sheath and RFEA system at various background gas pressures and radio-frequencies. 


\nocite{*}
\bibliography{references}

\end{document}
