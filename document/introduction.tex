\section{\label{Introduction}Introduction}

Retarding field analyzers are employed to assess the energy distribution of ions impinging on a surface. Ions acquire considerable kinetic energy as a result of the high voltage across the plasma sheath. This energy distribution is valuable in various applications, such as space propulsion equipment. In other contexts, such as the manufacturing of semiconductor devices, plasma etching equipment, or plasma deposition, the ion energy serves as a parameter that can influence the etch rate, the profile of the process, or the deposition rate.

RFEA devices can be designed as an array of thin grids tightly packed together, with a series of voltages applied to these grids. The primary purpose of this configuration is to select ions based on their energy and to eliminate the detection of electrons originating from the plasma discharge or secondary electrons emitted by the RFEA following ion impact. 

The operational range of RFEA devices is limited by several factors, including background gas pressure, fast electrons from the plasma, secondary electrons emitted within the RFEA, ionization occurring inside the RFEA, the distance between RFEA grids, and field distortion within the RFEA caused by space charge.

To assess the impact of any of the factors mentioned above, a computer simulation of a plasma sheath and RFEA grid would enable one to explore these factors individually or in combination.

Here we report a one-dimensional model of a plasma sheath and RFEA device. 
